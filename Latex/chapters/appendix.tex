

\begin{lstlisting}[language=Python]

#Bifurcation diagram

from cmath import inf
import numpy as np
import matplotlib.pyplot as plt
import math
import csv
import matplotlib as mpl
import time

start_time = time.time()
pos = 0
start=0
finish=4
step=0.0001
N=1001
k=np.zeros(len(range(0,N)))
x = np.zeros(len(range(0,N)))
i = np.zeros(len(range(0,N)))
M=np.zeros((40000,130))
q=-0.5
x[0]=1
g=12
#the path where the plots are saved

filename="./Latex/LateX images/sine q="+ str(q)+"/g" + str(g) +".jpg"
#filename="./Latex/LateX images/graphs q21/g" + str(g) +".jpg"
#filename="./Latex/LateX images/cheb q="+ str(q)+"/g" + str(g) +".jpg"

#the path where the the data of parameter k and x are saved

#file_path="./data3/ok q=" + str(q) +" x="+str(x[1])+".txt"

for k in np.arange(start,finish,step):
for i in range(1,N):
#np.sine - np.sinh
x[i] = k * math.sin(k* math.sinh(q * math.sin(2* x[i - 1]))); 
#x[i] = k *(1 + x[i - 1]) * (1 + x[i - 1]) * (2 - x[i - 1]) + q  #logistic
#x[i] = math.cos(k**q * math.acos(q*x[i - 1])) #cheb  
M[pos,:]=x[-130:]
pos += 1

#code for plotting the bif diagram

k=np.arange(start,finish,step)


plt.figure()
font = {'size': 45}
plt.rc('font', **font) 
for i in range(0,130):
plt.plot(k, M[:,i] ,'.k',alpha=1,ms=1.2)
plt.xlim(3.8,4)
plt.ylim(3,4 )
plt.rcParams.update({'text.usetex' : True})
plt.xlabel("k")
plt.ylabel("x")
figure = plt.gcf()  # get current figure
figure.set_size_inches(1920/40, 1080/40)
plt.savefig(filename,dpi=40)
print("--- %s seconds ---" % (time.time() - start_time))
plt.show()


exit()
#the code for saving the data
with open(file_path, 'w+',encoding='utf-8', newline='') as f:
for i in range(10000):
for j in range(130):
if np.any(M[i,j]==np.inf) or np.any(M[i,j]==-np.inf):
break
else:
f.writelines([f"{k[i]}",f"{M[i,j]}\n"])

f.close()
print("--- %s seconds ---" % (time.time() - start_time))




\end{lstlisting}

\begin{lstlisting}[language=Python]
#Lyapunov diagram

import numpy as np
import math
import matplotlib.pyplot as plt
import csv
import time
start_time = time.time()
q=0.9
g=2


#filename="./Latex/LateX images/sine q="+ str(q)+"/g" + str(g) +".jpg"
#filename="./Latex/LateX images/graphs q21/g" + str(g) +".jpg"
filename="./Latex/LateX images/cheb q="+ str(q)+"/g" + str(g) +".jpg"

start=0
finish=4.4
dim=0.00044
N=1001
def LE(start, finish , dim ):
e = 0.000000001
r=np.arange(start, finish ,dim )
x = np.zeros(len(range(0,N)))
x[0] =1
le2=np.zeros(len(range(0,len(r))))
for j in range(1,len(r)):
le=0
for i in range(1,N):

#x[i] = r[j] * math.sin(r[j]* math.sinh(q * math.sin(2* x[i - 1])))
#x[i] = r[j] * (1 + x[i - 1]) * (1 + x[i - 1]) * (2 - x[i - 1]) +q
x[i] = math.cos(r[j]**q * math.acos(q*x[i - 1]))  #cheb
x1 = np.zeros(len(range(0,N)))
x1[0]=x[N-1]
x2 = np.zeros(len(range(0,N)))
x2[0]=x[N-1]
for i in range(1,N):
# x1[i] = r[j] * math.sin(r[j]* math.sinh(q * math.sin(2* x1[i - 1])))
# x2[i - 1] = x1[i - 1] + e
# x2[i] = r[j] * math.sin(r[j]* math.sinh(q * math.sin(2* x2[i - 1])))

# x1[i] = r[j] * (1 + x1[i - 1]) * (1 + x1[i - 1]) * (2 - x1[i - 1]) +q
# x2[i - 1] = x1[i - 1] + e
# x2[i] = r[j] * (1 + x2[i - 1]) * (1 + x2[i - 1]) * (2 - x2[i - 1]) +q

x1[i] = math.cos(r[j]**q * math.acos(q*x1[i - 1]))
x2[i - 1] = x1[i - 1] + e
x2[i] = math.cos(r[j]**q * math.acos(q*x2[i - 1]))

dist=abs(x1[i]-x2[i])
if dist>0:
le = le + math.log(dist/e)

le2[j]=le/(N-1)
return(le2)

#code for plotting the Lyapunov diagram

LE=LE(start,finish,dim)
LElist = list(LE)
# for i in range(len(LElist)-1, 0, -1):
#     if LElist[i] < -6:
#         LElist.pop(i)
#         i=i+1 
k=np.arange(start,finish,dim)
font = {'size': 45}
plt.rc('font', **font) 
plt.figure()
plt.plot(k,LElist,'.k',alpha=1,ms=1.4)
plt.axhline(0)
plt.xlabel("k")
plt.ylabel("LE")
plt.xlim(0,4.4)
figure = plt.gcf()  # get current figure
figure.set_size_inches(1920/40, 1080/40)
plt.savefig(filename,dpi=40)
print("--- %s seconds ---" % (time.time() - start_time))
#plt.show()



\end{lstlisting}

\begin{lstlisting}[language=Python]
#x-x_i diagram

import numpy as np
import matplotlib.pyplot as plt
import math
from array import *

q=0.9
k=2.741
g=10

#the path where the plots are saved

#filename="./Latex/LateX images/sine q="+ str(q)+"/g" + str(g) +".png"
#filename="./Latex/LateX images/graphs q19/" + str(k) +".png"
filename="./Latex/LateX images/cheb q="+ str(q)+"/g" + str(g) +".png"

N=10**6+1
x = np.zeros(len(range(0, N)))
y= np.zeros(len(range(0, N)))
x[0] = 0
x[1] = 0.1
for i in range(2,N):
#np.sine - np.sinh
#x[i] = k * math.sin(k* math.sinh(q * math.sin(2* x[i - 1])));
#x[i] =k *(1 + x[i - 1]) * (1 + x[i - 1]) * (2 - x[i - 1]) + q #logistic
x[i] = math.cos(k**q * math.acos(q*x[i - 1]))  #cheb

#code for plotting x-x_i diagram   

xpoints=x[300:-1]
ypoints=x[301:]
plt.plot(xpoints,ypoints,'.',color='black',markersize=1)
plt.xlabel("x(i)")
plt.ylabel("x(i+1)")
plt.savefig(filename,bbox_inches='tight')
plt.show()
\end{lstlisting}
\clearpage
\begin{lstlisting}[language=Python]

#Code for robot path and coverage

import numpy as np
import math
import time
import matplotlib.pyplot as plt
start_time = time.time()
N=10**5
k=0.75
q=-1.4
g=4
#filename="./Latex/LateX images/log/xy/g" +str(g)+ str(q)+".jpg"
x =np.zeros(len(range(0, N)))
y = np.zeros(len(range(0, N)))
X = np.zeros(len(range(0, N)))
Y =np.zeros(len(range(0, N)))
M=np.zeros(len(range(0, N)))
theta=np.zeros(len(range(0, N)))


#chaotic maps
def rs(x1,y1):
#choose parameters for the two chaotic maps used

x[0]=x1
y[0]=y1
#choose parameters for the robot

X[0]=0
Y[0]=0
#theta[1]=0
L= 0.15
h=0.2

for i in range(1,N):
x[i] =k *((1 + x[i - 1]) *(1 + x[i - 1])) * (2 - x[i - 1]) +q
y[i] =k *((1 + y[i - 1]) *(1 + y[i - 1]))  * (2 - y[i - 1]) +q

#robot coordinates
X[i]=X[i-1]+h*math.cos(theta[i-1])*(x[i-1]+y[i-1])/2
Y[i]=Y[i-1]+h*math.sin(theta[i-1])*(x[i-1]+y[i-1])/2
theta[i]=theta[i-1]+h*(x[i-1]-y[i-1])/L
#keep the robot inside the boundaries
if X[i]>=40 or X[i] <=0:
X[i]=X[i-1]-h*math.cos(theta[i-1])*(x[i-1]+y[i-1])/2
if Y[i]>=40 or Y[i] <=0 :
Y[i]=Y[i-1]-h*math.sin(theta[i-1])*(x[i-1]+y[i-1])/2
return(X,Y)

rs1=rs(0.5,1)
# rs2=np.array(rs1, dtype=np.int)
plt.figure()
font = {'size': 45}
plt.rc('font', **font) 
plt.plot(X,Y ,'--b',alpha=0.8,ms=1)
plt.rcParams.update({'text.usetex' : True})
plt.rcParams['agg.path.chunksize'] = 10000
plt.xlim(0,40)
plt.ylim(0,40)
plt.xlabel("X")
plt.ylabel("Y")
figure = plt.gcf()  # get current figure
figure.set_size_inches(1920/40, 1080/40)
#plt.savefig(filename,dpi=40)
print("--- %s seconds ---" % (time.time() - start_time))
#plt.show()

#create a zero matrix of appropriate dimansions
#assuming each cell is 0.25x0.25
#cell=40/0.25
I= np.zeros((len(np.arange(0,160)),)*2)
#print(len(I))

#I=np.zeros(cell,cell,dtype=float)
# gridx=math.floor(X[1]/0.25)+1
# gridy=math.floor(Y[1]/0.25)+1
# print((gridx))
# print((gridy))

# I[gridx,gridy]=1
# print(len(I[gridx,gridy]))
#X[1]=0
#Y[1]=0
#print(X)

for j in range(1,len(X)):
#gia kathe sintetagmeni X,Y, ipologizw to cell (keli) poy antistoixei 
# sto (X,Y)(i), kai epeita sta mikos 1/3, 1/2 kai 2/3 toy diastimatos metaksi
#(X,Y)(i) kai (X,Y)(i-1). etsi px an se kapoio iteration to robot
#kanei megalo 'alma', na ipologistoun k ta endiamesa cells
if X[j]>=0 or X[j]<=40 or Y[j]>=0 or Y[j]<=40:

gridx=math.floor(X[j]/0.25)
gridy=math.floor(Y[j]/0.25)
I[gridx,gridy]=1

#2/3
gridx=math.floor((0.3*X[j-1]+0.7*X[j])/0.25)
gridy=math.floor((0.3*Y[j-1]+0.7*Y[j])/0.25)
I[gridx,gridy]=1
#1/3
gridx=math.floor((0.7*X[j-1]+0.3*X[j])/0.25)
gridy=math.floor((0.7*Y[j-1]+0.3*Y[j])/0.25)
I[gridx,gridy]=1
#1/2
gridx=math.floor((0.5*X[j-1]+0.5*X[j])/0.25)
gridy=math.floor((0.5*Y[j-1]+0.5*Y[j])/0.25)
I[gridx,gridy]=1
coverage=(np.mean((np.mean(I)*100)))

print(coverage)

print("--- %s seconds ---" % (time.time() - start_time))

\end{lstlisting}

\begin{lstlisting}[language=Python]

#Code for plotting steps and coverage

import numpy as np
import matplotlib.pyplot as plt
import matplotlib as mpl
g=2
filename="./Latex/LateX images/log/steps/g" +str(g)+".jpg"
y1=[10.46,19.58,23.2,27.57,51.25,63.78]
x1=[10**5,5*10**5,7*10**5,10**6,5*10**6,10**7]
y2= [15.125,54.254,64.840,81.082,99.98,100]
x2=[10**4,5*10**4,7*10**4,10**5,5*10**5,10**6]

font = {'size': 45}
plt.rc('font', **font)
mpl.rc('lines', linewidth=8, linestyle='solid')
plt.plot(x2,y2) 
plt.scatter(x2,y2 , s=600, zorder=2.5)
plt.rcParams.update({'text.usetex' : True})
plt.rcParams['agg.path.chunksize'] = 10000 
plt.xlabel("Αριθμός Βημάτων")
plt.ylabel("Ποσοστό Κάλυψης (%)")
figure = plt.gcf()  # get current figure
figure.set_size_inches(1920/40, 1080/40)
plt.savefig(filename,dpi=40)
#print("--- %s seconds ---" % (time.time() - start_time))
plt.show()
\end{lstlisting}

\begin{thebibliography}{9}
	\bibitem{b1}
	Αναστάσιος Μπούντης \emph{Δυναμικά Συστήματα και Χάος}, Εκδόσεις Παπασωτηρίου, 1995
	
	\bibitem{b2}
	Βουγιατζής, Γ., Μελετλίδου, Ε.,\emph{ Εισαγωγή στα μη γραμμικά δυναμικά συστήματα}, Σύνδεσμος
	Ελληνικών Ακαδημαϊκών Βιβλιοθηκών, 2015
	
	\bibitem{b3}
	Παύλος, Γ.Π., \emph{Ντετερμινιστικά Συστήματα – Στοιχεία Χαοτικής Ανάλυσης Χρονοσειρών}, 
	\url{http://utopia.duth.gr/~gpavlos/deterministic_systems.pdf}
	
	\bibitem{b4}
	Σουρλάς, Δ., \emph{Δυναμικά Συστήματα και Εφαρμογές με τη χρήση του Maple}, Πανεπιστήμιο Πατρών Τμήμα Φυσικής, 2010
	
	\bibitem{b5}
	Κυπριανίδης, Γ., Πετράνη, Μ., \emph{ΜΗ ΓΡΑΜΜΙΚΑ ΚΥΚΛΩΜΑΤΑ από την περιοδική στη χαοτική συμπεριφορά}, Εκδόσεις Σύγχρονη Παιδεία, 2008
	\bibitem{b6}
	Δήμητρα Βλόντζου, \emph{Σχεδίαση Διαδρομής Αυτόνομου Ρομποτικού Οχήματος με Χρήση Διακριτού Χαοτικού Συστήματος}, \url{https://ikee.lib.auth.gr/record/335179/files/VLONTZOU.pdf}
	\bibitem{b7}
	Χρυσάνθη Τσιάρα, \emph{Μελέτη Χαοτικών Χαρτών και Χρήση τους σε
		Συστήματα Κρυπτογραφίας Εικόνας},\url{https://ikee.lib.auth.gr/record/335142/files/Tsiara.pdf}
	\bibitem{b8}
	Moysis, Lazaros and Petavratzis, Eleftherios and Volos, Christos and Nistazakis, Hector and Stouboulos, Ioannis, \emph{A chaotic path planning generator based on logistic map and modulo tactics}, Robotics and Autonomous Systems, 11/2019
	
	
\end{thebibliography}


\clearpage