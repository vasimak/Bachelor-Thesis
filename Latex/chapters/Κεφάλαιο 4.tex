Στο κεφάλαιο αυτό παρουσιάζεται αναλυτικά η μελέτη της δυναμικής συμπεριφοράς ενός διακριτού συστήματος που αποτελεί παραλλαγή του γνωστού Chebysev Χάρτη. Για επιλεγμένες τιμές της παραμέτρου του μποροεί να παρουσιάσει χαοτική συμπεριφορά όπως και φαινόμενα που σχετίζονται με τη μη-γραμμική δυναμική. Για την μελέτη χρησιμοποιήθηκαν τα διαγράμματα διακλάδωσης, οι εκθέτες Lyapunov και οι απεικονίσεις της τιμής \(x_i\) σε συνάρτηση με  την τιμή \(x_{i+1}\).\\\\

Ο Chebysev Χάρτης που αποτέλεσε τη βάση του προτεινόμενου σε αυτή την ενότητα, χάρτη, περιγράφεται από την παρακάτω εξίσωση:

\begin{equation}
	x_i=\cos{k*\arccos{x_{i-1}}}
	\label{f:x4}
\end{equation}


Στην εξίσωση (\ref{f:x4}) προστέθηκε ένας σταθερός όρος \emph{q}. Έτσι προέκυψε η προτεινόμενη παραλλαγή του Λογιστικού Χάρτη,

\begin{equation}
	x_i=\cos{k^q\arccos{q*x_{i-1}}}
	\label{f:x5}
\end{equation}
όπου \emph{k}, \emph{q} : παράμετροι.\\

Για την εύρεση της δυναμικής συμπεριφοράς του συστήματος εξετάστηκε μια περιοχή τιμών των συγκεκριμένων παραμέτρων, ώστε να επιτευχθεί ταυτόχρονη σύγκριση της περιοδικής και χαοτικής συμπεριφοράς του. Πιο συγκεκριμένα, στη μελέτη που πραγματοποιήθηκε η αρχική συνθήκη του $x_0 =0.1$ παρέμεινε  σταθερή, ενώ η τιμή της παραμέτρου \emph{q} μεταβαλλόταν στο διάστημα $[0.8,-0.9]$ με βήμα $0.1$. Έτσι, για κάθε περίπτωση παράχθηκαν το διάγραμμα διακλάδωσης, το διάγραμμα των εκθετών Lyapunov και το διάγραμμα της τιμής \(x_i\) σε συνάρτηση με  την τιμή \(x_{i+1}\), τα οποία παρουσιάζονται και αναλύονται στη συνέχεια.\\




\section{Για q = 0.8}

Στo Σχ. \ref{f:g59} παρατίθεται τα διάγραμμα διακλάδωσης του συστήματος \ref{f:x5}, ως προς την παράμετρο \emph{k}, για $q =0.8$. Στον πίνακα \ref{tab:abc12} φαίνεται η πορεία του συστήματος και για ποιες τιμές της παραμέτρου \emph{k} το σύστημα εμφανίζει περιοδική ή χαοτική συμπεριφορά, σύμφωνα με το διάγραμμα διακλάδωσης \ref{f:g44}. Επίσης παρατηρείται εσωτερική κρίση ελκυστών για διάφορες τιμές του \emph{k} (2.65, 2.938, 3.147, 3.45, 3.642, 3.776, 3.886, 4.1, 4.155), όπως και το φαινόμενο της υστέρησης το οποίο φαίνεται στο διάγραμμα διακλάδωσης \ref{f:g44}. Οι αντίστοιχες τιμές του \emph{k} για αυτά τα σημεία του διαγράμματος υπάρχουν στο πίνακα \ref{tab:abc12}, όπως και τα αντίστοιχα σχήματα \ref{f:k248}, \ref{f:k249} των διαγραμμάτων της τιμής \(x_i\) σε συνάρτηση με την τιμή \(x_{i+1}\). Από τα παραγόμενα σχήματα προκύπτει αριθμός σημείων αντίστοιχος με την περίοδο του συστήματος.

Επιπλέον παρατηρούμε στo Σχ. \ref{f:g64}, το φαινόμενο της αντιμονοτονικότητας. Στα τρία διαγράμματα \ref{f:g62}, \ref{f:g63} \ref{f:g64} εμφανίζεται μία χαοτική φυσαλίδα (το σύστημα εισέρχεται στο χάος με διπλασιασμό της περιόδου και στην συνέχεια εξέρχεται από αυτό με αντίστροφο διπλασιασμό της περιόδου.). Συγκεκριμένα στο διάγραμμα \ref{f:g64} παρατηρούμε ότι ενδιάμεσα του βασικού ορθού διπλασιασμόυ εμφανίζεται ένας ακόμα υπο την μορφή εσωτερικής κρίσης για $k=3.886$ και $k=4.1$ Ακόμη στο διάγραμμα \ref{f:g63} το φαινόμενο εμφανίζεται άλλη μία φορά για $3.17<k<3.258$, όπου παρατηρείται ότι για $k=3.17$ εμφανίζεται ένας διπλασιασμός (\emph{περίοδος - 10}) ο οποίος καταστρέφεται για $k=3.258$.

Τέλος, στο σχήμα \ref{f:g60} παρατίθεται το διάγραμμα των εκθετών Lyapunov για τιμές του \emph{k} στο ίδιο διάστημα τιμών $[0, 4.4]$. Οι τιμές του πίνακα \ref{tab:abc12} που έχουνε περιοδική συμπεριφορά αντιστοιχούν σε τιμές του διαγράμματος \ref{f:g59} όπου o εκθέτης Lyapunov είναι συνεχώς αρνητικός, γεγονός που επιβεβαιώνει την συμπεριφορά του. Ενώ για τις υπόλοιπες τιμές ο θετικός εκθέτης Lyapunov υποστηρίζει την χαοτική του συμπεριφορά, όπως έγινε φανερό και από το διάγραμμα διακλάδωσης.\\\\

\begin{table}[ht]
	\centering
	\caption{ Συμπεριφορά του υπό μελέτη συστήματος για διάφορες τιμές του \emph{k}, για $q=0.8$ }
	\label{tab:abc12}
	\begin{tabular}{l | l}
		Παράμετρος k & Συμπεριφορά \\
		\hline
		1.3 &  Περίοδος -  1 \\
		1.86 &  Περίοδος -  2 \\
		2.34& Περίοδος -  4 \\
		2.49& Περίοδος -  8 \\
		2.52& Περίοδος -  16 \\
		2.53 & Xάος \\
		2.65& Περίοδος - 6 \\
		2.655& Περίοδος - 12\\
		2.66& Χάος \\
		2.938& Περίοδος - 5 \\
		2.95 &  Περίοδος - 10  \\
		2.971 &  Περίοδος -  20 \\
		2.975 &  Χάος \\
		3.147& Περίοδος - 20 \\
		3.15 &  Περίοδος - 10  \\
		3.17 &  Περίοδος -  5 \\
		3.24 &Περίοδος - 10 \\
		3.258 &  Περίοδος -  5\\
		3.28 &Χάος \\
		3.45 & Περίοδος - 4\\
		3.453& Περίοδος - 8\\
		3.455& Περίοδος - 16\\
		3.46& Xάος\\
		3.642& Περίοδος - 16\\
		3.643 & Περίοδος - 8\\
		3.647& Περίοδος - 4\\
		3.65 & Χάος\\
		3.776 & Περίοδος -  2\\
		3.82 & Περίοδος -  4\\
		3.85 & Περίοδος -  8\\
		3.86 & Xάος\\
		3.886 & Περίοδος -  6\\
		3.887 & Περίοδος -  12\\
		3.888 & Χάος\\
		4.1& Περίοδος -  24\\
		4.101& Περίοδος -  12\\
		4.102 & Περίοδος -  6\\
		4.108 & Χάος\\
		4.155 & Περίοδος -  8\\
		4.17 & Περίοδος -  4\\
		4.22 & Περίοδος -  2\\
		4.32 &  Χάος\\
			
	\end{tabular}
	
\end{table}

\begin{figure}[ht]
	\centering
	\includegraphics[width=1\linewidth]{LateX images/cheb q=0.8/g1}
	\caption{Διάγραμμα διακλάδωσης, για $q=0.8$.}
	\label{f:g59}
\end{figure}


\begin{figure}[ht]
	\centering
	\includegraphics[width=1\linewidth]{LateX images/cheb q=0.8/g2}
	\caption{Διάγραμμα των εκθετών Lyapunov σε συνάρτηση με την παράμετρο \emph{k}, για $q=0.8$.}
	\label{f:g60}
\end{figure}


\begin{figure}[ht]
	\centering
	
	\begin{subfigure}[b]{0.8\textwidth}
		\centering
		\includegraphics[width=\textwidth]{LateX images/cheb q=0.8/g3}
		\caption{Για $2.9<k<3.3$}
		\label{f:g61}
	\end{subfigure}
	\hfill
	\begin{subfigure}[b]{0.8\textwidth}
		\centering
		\includegraphics[width=\textwidth]{LateX images/cheb q=0.8/g4}
		\caption{Για $3.4<k<3.7$}
		\label{f:g62}
	\end{subfigure}
	\hfill
	\begin{subfigure}[b]{0.8\textwidth}
		\centering
		\includegraphics[width=\textwidth]{LateX images/cheb q=0.8/g5}
		\caption{Για $3.6<k<4.7$}
		\label{f:g63}
	\end{subfigure}

	\caption{Διαγράμματα διακλάδωσης για διάφορες τιμές του $k$. }
	\label{f:g64}
\end{figure}

\begin{figure}[ht]
	\centering
	\begin{subfigure}[b]{0.4\textwidth}
		\centering
		\includegraphics[width=\textwidth]{LateX images/cheb q=0.8/g6}
		\caption{Για $k=2.65$}
		\label{f:k133}
	\end{subfigure}
	\hfill
	\begin{subfigure}[b]{0.4\textwidth}
		\centering
		\includegraphics[width=\textwidth]{LateX images/cheb q=0.8/g7}
		\caption{Για $k=2.938$}
		\label{f:k134}
	\end{subfigure}
	\hfill
	\begin{subfigure}[b]{0.4\textwidth}
		\centering
		\includegraphics[width=\textwidth]{LateX images/cheb q=0.8/g8}
		\caption{Για $k=3.147$}
		\label{f:k135}
	\end{subfigure}
	\hfill
	\begin{subfigure}[b]{0.4\textwidth}
		\centering
		\includegraphics[width=\textwidth]{LateX images/cheb q=0.8/g9}
		\caption{Για $k=3.45$}
		\label{f:k136}
	\end{subfigure}
	\hfill	
	\caption{Διαγράμματα της τιμής \(x_i\) σε συνάρτηση με την τιμή \(x_{i+1}\) (α' μέρος).}
	\label{f:k248}
\end{figure}
\begin{figure}[ht]
	\centering
		\begin{subfigure}[b]{0.4\textwidth}
		\centering
		\includegraphics[width=\textwidth]{LateX images/cheb q=0.8/g10}
		\caption{Για $k=3.642$}
		\label{f:k137}
	\end{subfigure}
	\hfill	
	\begin{subfigure}[b]{0.4\textwidth}
		\centering
		\includegraphics[width=\textwidth]{LateX images/cheb q=0.8/g11}
		\caption{Για $k=3.776$}
		\label{f:k138}
	\end{subfigure}
	\hfill
	\begin{subfigure}[b]{0.4\textwidth}
		\centering
		\includegraphics[width=\textwidth]{LateX images/cheb q=0.8/g12}
		\caption{Για $k=3.886$}
		\label{f:k139}
	\end{subfigure}
	\hfill
	\begin{subfigure}[b]{0.4\textwidth}
		\centering
		\includegraphics[width=\textwidth]{LateX images/cheb q=0.8/g13}
		\caption{Για $k=4.1$}
		\label{f:k140}
	\end{subfigure}
	\hfill
	\begin{subfigure}[b]{0.4\textwidth}
		\centering
		\includegraphics[width=\textwidth]{LateX images/cheb q=0.8/g14}
		\caption{Για $k=4.155$}
		\label{f:k141}
	\end{subfigure}
	\hfill
	\caption{Διαγράμματα της τιμής \(x_i\) σε συνάρτηση με την τιμή \(x_{i+1}\) (β' μέρος).}
	\label{f:k249}
\end{figure}

\clearpage

\section{Για q = 0.9}

Στo Σχ. \ref{f:g65} παρατίθεται τα διάγραμμα διακλάδωσης του συστήματος \ref{f:x5}, ως προς την παράμετρο \emph{k}, για $q =0.9$. Στον πίνακα \ref{tab:abc13} φαίνεται η πορεία του συστήματος και για ποιες τιμές της παραμέτρου \emph{k} το σύστημα εμφανίζει περιοδική ή χαοτική συμπεριφορά, σύμφωνα με το διάγραμμα διακλάδωσης \ref{f:g65}. Επίσης παρατηρείται εσωτερική κρίση ελκυστών για διάφορες τιμές του \emph{k} (1.96, 2.015, 2.16, 2.319, 2.603, 2.638, 2.725, 2.773), όπως και το φαινόμενο της υστέρησης το οποίο φαίνεται στο διάγραμμα διακλάδωσης \ref{f:g65}. Οι αντίστοιχες τιμές του \emph{k} για αυτά τα σημεία του διαγράμματος υπάρχουν στο πίνακα \ref{tab:abc13}, όπως και τα αντίστοιχα σχήματα \ref{f:k250}, \ref{f:k251} των διαγραμμάτων της τιμής \(x_i\) σε συνάρτηση με την τιμή \(x_{i+1}\). Από τα παραγόμενα σχήματα προκύπτει αριθμός σημείων αντίστοιχος με την περίοδο του συστήματος.

Επιπλέον παρατηρούμε στo Σχ. \ref{f:g67}, το φαινόμενο της αντιμονοτονικότητας. Συγκεκριμένα εμφανίζεται μία χαοτική φυσαλίδα (το σύστημα εισέρχεται στο χάος με διπλασιασμό της περιόδου και στην συνέχεια εξέρχεται από αυτό με αντίστροφο διπλασιασμό της περιόδου.) για $2.603<k<2.647$. Ακόμη στο διάγραμμα \ref{f:g67} το φαινόμενο εμφανίζεται άλλη μία φορά για $2.725<k<2.732$, όπου παρατηρείται ότι για $k=2.727$ εμφανίζεται ένας διπλασιασμός (\emph{περίοδος - 18}) ο οποίος καταστρέφεται για $k=2.732$.
Επίσης παρατηρούμε ότι μεταξύ των δύο βασικών ορθών διπλασιασμών εμφανίζεται ένας αντίστροφος για $2.638<k<2.71$.

Τέλος, στο σχήμα \ref{f:g66} παρατίθεται το διάγραμμα των εκθετών Lyapunov για τιμές του \emph{k} στο ίδιο διάστημα τιμών $[0, 3]$. Οι τιμές του πίνακα \ref{tab:abc13} που έχουνε περιοδική συμπεριφορά αντιστοιχούν σε τιμές του διαγράμματος \ref{f:g64} όπου o εκθέτης Lyapunov είναι συνεχώς αρνητικός, γεγονός που επιβεβαιώνει την συμπεριφορά του. Ενώ για τις υπόλοιπες τιμές ο θετικός εκθέτης Lyapunov υποστηρίζει την χαοτική του συμπεριφορά, όπως έγινε φανερό και από το διάγραμμα διακλάδωσης.\\\\

\begin{table}[ht]
	\centering
	\caption{ Συμπεριφορά του υπό μελέτη συστήματος για διάφορες τιμές του \emph{k}, για $q=0.9$ }
	\label{tab:abc13}
	\begin{tabular}{l | l}
		Παράμετρος k & Συμπεριφορά \\
		\hline
		1.3 &  Περίοδος -  1 \\
		1.5 &  Περίοδος -  2 \\
		1.8& Περίοδος -  4 \\
		1.91& Περίοδος -  8 \\
		1.94& Περίοδος -  16 \\
		1.95 & Xάος \\
		1.96& Περίοδος - 12 \\
		1.97& Xάος \\
		2.014& Περίοδος - 6 \\
		2.019& Περίοδος - 12\\
		2.02& Χάος \\
		2.164& Περίοδος - 5 \\
		2.169 &  Περίοδος - 10  \\
		2.17 &  Χάος \\
		2.319& Περίοδος - 3 \\
		2.355 &  Περίοδος - 6  \\
		2.375 &  Περίοδος -  12 \\
		2.38 &Χάος \\
		2.603 & Περίοδος - 6\\
		2.61& Περίοδος - 12\\
		2.621& Περίοδος - 24\\
		2.623& Xάος\\
		2.638 & Περίοδος - 24\\
		2.639& Περίοδος - 12\\
		2.647& Περίοδος - 6\\
		2.66 & Περίοδος - 3\\
		2.7 & Περίοδος -  6\\
		2.71 & Περίοδος -  12\\
		2.715 & Xάος\\
		2.725 & Περίοδος - 9\\
		2.727 & Περίοδος -  18\\
		2.732 & Περίοδος -  9\\
		2.738 & Περίοδος -  24\\
		2.74& Χάος\\
		2.773 & Περίοδος -  6\\
		2.774& Χάος\\
		
	\end{tabular}
	
\end{table}


\begin{figure}[ht]
	\centering
	\includegraphics[width=1\linewidth]{LateX images/cheb q=0.9/g1}
	\caption{Διάγραμμα διακλάδωσης, για $q=0.9$.}
	\label{f:g65}
\end{figure}


\begin{figure}[ht]
	\centering
	\includegraphics[width=1\linewidth]{LateX images/cheb q=0.9/g2}
	\caption{Διάγραμμα των εκθετών Lyapunov σε συνάρτηση με την παράμετρο \emph{k}, για $q=0.9$.}
	\label{f:g66}
\end{figure}


\begin{figure}[ht]
	\centering
	\includegraphics[width=\textwidth]{LateX images/cheb q=0.9/g3}
	\label{f:g67}
	\caption{Διαγράμματα διακλάδωσης για $2.55<k<2.8$ }
\end{figure}

\begin{figure}[ht]
	\centering
	\begin{subfigure}[b]{0.4\textwidth}
		\centering
		\includegraphics[width=\textwidth]{LateX images/cheb q=0.9/g4}
		\caption{Για $k=1.96$}
		\label{f:k142}
	\end{subfigure}
	\hfill
		\begin{subfigure}[b]{0.4\textwidth}
		\centering
		\includegraphics[width=\textwidth]{LateX images/cheb q=0.9/g5}
		\caption{Για $k=2.014$}
		\label{f:k143}
	\end{subfigure}
	\hfill
	\begin{subfigure}[b]{0.4\textwidth}
		\centering
		\includegraphics[width=\textwidth]{LateX images/cheb q=0.9/g6}
		\caption{Για $k=2.16$}
		\label{f:k144}
	\end{subfigure}
	\hfill
	\begin{subfigure}[b]{0.4\textwidth}
		\centering
		\includegraphics[width=\textwidth]{LateX images/cheb q=0.9/g7}
		\caption{Για $k=2.319$}
		\label{f:k145}
	\end{subfigure}
	\hfill

	\caption{Διαγράμματα της τιμής \(x_i\) σε συνάρτηση με την τιμή \(x_{i+1}\) (α' μέρος).}
	\label{f:k250}
\end{figure}
\begin{figure}[ht]	
	\centering
		\begin{subfigure}[b]{0.4\textwidth}
		\centering
		\includegraphics[width=\textwidth]{LateX images/cheb q=0.9/g8}
		\caption{Για $k=2.603$}
		\label{f:k146}
	\end{subfigure}
	\hfill
	\begin{subfigure}[b]{0.4\textwidth}
		\centering
		\includegraphics[width=\textwidth]{LateX images/cheb q=0.9/g9}
		\caption{Για $k=2.637$}
		\label{f:k147}
	\end{subfigure}
	\hfill	
	\begin{subfigure}[b]{0.4\textwidth}
		\centering
		\includegraphics[width=\textwidth]{LateX images/cheb q=0.9/g10}
		\caption{Για $k=2.725$}
		\label{f:k148}
	\end{subfigure}
	\hfill	
	\begin{subfigure}[b]{0.4\textwidth}
		\centering
		\includegraphics[width=\textwidth]{LateX images/cheb q=0.9/g11}
		\caption{Για $k=2.773$}
		\label{f:k149}
	\end{subfigure}
	\hfill

	\caption{Διαγράμματα της τιμής \(x_i\) σε συνάρτηση με την τιμή \(x_{i+1}\) (β' μέρος).}
	\label{f:k251}
\end{figure}

\clearpage