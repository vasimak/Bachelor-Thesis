

Η Python τα τελευταία χρόνια χρησιμοποιείται όλο και περισσότερο στον κόσμο του προγραμματισμού  και κυρίως στους επιστημονικούς κύκλους. Δίνεται η επιλογή στον ερευνητή να χρησιμοποιήσει απείρες βιβλιοθήκες οι οποίες είναι κατάλληλες ακόμα και για τα πιο εξιδεικευμένα έργα. Είναι αρκετά εύπλαστη, ένα προτέρημα που την κάνει να ξεχωρίζει από παλιότερες γλώσσες προγραμματισμού. Επίσης όλα τα παραπάνω την κάνουν αρκετά προσιτή σε οποιοδήποτε άτομο θέλει να ξεκινήσει να ασχολείται με τον προγραμματισμό, χωρίς να έχει προηγούμενη εμπειρία στο αντικείμενο.

Οι κώδικες που γράφτηκαν για την μελέτη τόσο των διακριτών χαοτικών συστημάτων, όσο και για την εφαρμογή τους στον έλεγχο της κίνησης του ρομποτικού οχήματος, παρατίθενται στη συνέχεια.

\lstinputlisting[language=Python,caption=Bifurcation Diagram]{chapters/python/Bifurcation diagram.py}

\lstinputlisting[language=Python,caption=Lyapunov expotent]{chapters/python/Lyapunov expotent.py}

\lstinputlisting[language=Python,caption=$x_i-x_i+1$ diagram]{chapters/python/x_i-x_i+1 diagram.py}

\lstinputlisting[language=Python,caption=Robot path coverage]{chapters/python/Robot_path_coverage.py}

\lstinputlisting[language=Python,caption=Steps coverage diagram]{chapters/python/Steps_coverage diagram.py}
