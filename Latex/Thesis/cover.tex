\begin{titlepage}
	\afterpage{\blankpage}	
	\begin{figure}[H]
		\begin{center}
			\includegraphics[width=5cm]{LateX images/auth}
			\label{fig:cover_auth_logo}
		\end{center}
	\end{figure}
	
	\centering
	\Large Αριστοτέλειο Πανεπιστήμιο Θεσσαλονίκης\\
	\Large Σχολή θετικών Επιστημών\\
	\large Τμήμα Φυσικής\\
	\large Εργαστήριο Μη-Γραμμικών Συστημάτων, Κυκλωμάτων και Πολυπλοκότητας
	
	\vspace{\fill}
	
	\LARGE Μελέτη Διακριτών Χαοτικών Συστημάτων και Εφαρμογή τους στον Έλεγχο της Κίνησης Αυτόνομων  Ρομποτικών Οχημάτων με τη Χρήση της Python
	
	\vspace{\fill}
	
	\Large Πτυχιακή Εργασία\\
	\Large του\\
	\Large Βασίλειου Ασημακόπουλου
	
	\vspace{\fill}
	\raggedright
	
	\begin{tabular}{ll}
		\textbf{Επιβλέπων:} & Χρήστος Βόλος\\
		& Αναπληρωτής Καθηγητής Α.Π.Θ.\\
	\end{tabular}
	
	\centering
	\vspace{\fill}
	\today
	
\end{titlepage}
\afterpage{\blankpage}
\begin{abstract}
	H παρούσα πτυχιακή εργασία ασχολείται με την υλοποίηση και μελέτη της δυναμικής συμπεριφοράς διακριτών χαοτικών συστημάτων με τη γλώσσα προγραμματισμού Python.   
	
	Για τον σκοπό αυτό χρησιμοποιήθηκαν παραλλαγές γνωστών μη - γραμμικών διακριτών συστημάτων που εμφανίζουν χαοτική συμπεριφορά και αναλύθηκαν τα φαινόμενα που παρατηρούνται με την μεταβολή διάφορων παραμέτρων τους. Γράφηκαν κώδικες που παράγουν τα διαγράμματα διακλάδωσης, τους εκθέτες Lyapunov και τα διαγράμματα των τιμών $x_i$ σε συνάρτηση με τις τιμές $x_{i+1}$. Επίσης, ως εφαρμογή αυτών των συστημάτων γράφηκαν κώδικες για τη μελέτη του ελέγχου της κίνησης αυτόνομων ρομποτικών οχημάτων μέσω των διακριτών συστημάτων που χρησιμοποιήθηκαν. 
	
\end{abstract}
\afterpage{\blankpage}
\begin{otherlanguage}{english} 
	\begin{abstract}
		This thesis deals with the implementation and study of the dynamic behavior of discrete chaotic systems with the Python programming language.
		
		For this purpose, variants of known nonlinear discrete systems that exhibit chaotic behavior have been used and the phenomena observed by varying their various parameters have been analyzed. Codes have been written that produce the bifurcation diagrams, Lyapunov exponents, and plots of $x_i$ values versus $x_{i+1}$ values. Also, as an implementation of these systems, codes have been written to study the control of movement of autonomous robotic vehicles by using the discrete systems.
		
		
	\end{abstract}
\end{otherlanguage}



\pagenumbering{gobble}
\chapter*{Πρόλογος} 


Η πτυχιακή αυτή εργασία αποτελεί μαι προσπάθεια να διερευνηθεί η χρήση της γλώσσας προγραμματισμού Python στην επίλυση κι μελέτη της δυναμικής συμπεριφοράς διακριτών χαοτικών συστημάτων. 

Ειδικότερα η πτυχιακή αποτελείται από πέντε κεφάλαια, όπου το κάθε ένα παρουσιάζει τη μελέτη παραλλαγών γνωστών διακριτών συστημάτων, ενώ επιπλέον στο τελευταίο κεφάλαιο παρουσιάζεται μια ενδιαφέρουσα εφαρμογή τους στο πεδίο της ρομποτικής. 

Συγκεκριμένα, στο πρώτο κεφάλαιο αναπτύσσεται το απαραίτητο θεωρητικό υπόβαθρο για την κατανόηση της μελέτης των δυναμικών συστημάτων. Ειδικότερα αναλύεται ο ορισμός των δυναμικών συστημάτων και η σχέση τους με το χάος. Ορίζονται τα εργαλεία που αξιοποιούνται στην παρούσα εργασία για την μελέτη των συστημάτων, όπως και τα φαινόμενα που παρατηρήθηκαν μέσα από αυτήν.

Στα επόμενα τρία κεφάλαια μελετούνται οι παραλλαγές τριών γνωστών μη - γραμμικών διακριτών δυναμικών συστημάτων. Συγκεκριμένα στο δεύτερο κεφάλαιο μελετάται η παραλλαγή του \emph{Λογιστικού χάρτη} και τα φαινόμενα που εμφανίζει μέσα από την παρατήρηση των διαγραμμάτων διακλάδωσης και των εκθετών Lyapunov. Στο τρίτο κεφάλαιο μελετάται η παραλλαγή του \emph{sine-sinh-sine χάρτη} και οι συμπεριφορές που εμφανίζει όσο μεταβάλλεται μία παράμετρος του. Τέλος και στο τέταρτο κεφάλαιο πραγματοποιείται αντίστοιχη μελέτη για την παραλλαγή του \emph{Chebysev χάρτη}.

Στο πέμπτο κεφάλαιο μελετήθηκε, μέσω προσομοίωσης, η κίνηση ενός αυτόνομου ρομποτικού οχήματος και η κάλυψη μιας συγκεκριμένης περιοχής του χώρου, με την αξιοποίηση του \emph{Λογιστικού χάρτη}, συναρτήσει διαφόρων παραμέτρων του. Η βασική λειτουργία του αυτόνομου ρομποτικού οχήματος παρουσιάζεται στην αρχή του κεφαλαίου .

Στο σημείο αυτό θα ήθελα να ευχαριστήσω τον επιβλέποντα της πτυχιακής, Αναπληρωτή Καθηγητή κ. Χρήστο Βόλο για τον χρόνο που αφιέρωσε για να μου απαντήσει σε οποιαδήποτε απορία είχα γύρω από το θέμα της πτυχιακής εργασίας, αλλά και την υπομονή που έδειξε μέσα σε αυτό το χρονικό διάστημα, μετατρέποντας την εργασία σε μία ευχάριστη εμπειρία.

Επίσης θα ήθελα να ευχαριστήσω δύο κοντινά μου άτομα για την υπομονή που δείξανε και τον χρόνο που αφιέρωσαν στο να με βοηθήσουν στον κώδικα που έγραψα, όπως και τους ανθρώπους του εργαστηρίου LaNSCom που ήταν εκεί για να απαντήσουν κάθε μου ερώτηση.

Τέλος θα ήθελα να ευχαριστήσω τους κοντινούς μου ανθρώπους, που χωρίς την συμπαράσταση τους, τα θερμά τους λόγια, και τις στιγμές που πίστευαν περισσότερο αυτοί σε εμένα, δεν θα μπορούσα να τελειώσω αυτή την εργασία.


