\begin{titlepage}
	
\begin{figure}[H]
		\begin{center}
			\includegraphics[width=3cm]{LateX images/auth}
			\label{fig:cover_auth_logo}
		\end{center}
	\end{figure}
	
	\centering
	\Large Αριστοτέλειο Πανεπιστήμιο Θεσσαλονίκης\\
	\Large Σχολή θετικών Επιστημών\\
	\large Τμήμα Φυσικής\\
	\large Εργαστήριο Μη-Γραμμικών Κυκλωμάτων, Συστημάτων και Πολυπλοκότητας
	
	\vspace{\fill}
	
	\LARGE Έλεγχος  της  Κίνησης  Αυτόνομων  Ρομποτικών  Οχημάτων  με  τη   Χρήση Χαοτικών Συστημάτων
	
	\vspace{\fill}
	
	\Large Πτυχιακή Εργασία\\
	\Large του\\
	\Large Βασίλειου Ασημακόπουλου
	
	\vspace{\fill}
	\raggedright
	
	\begin{tabular}{ll}
		\textbf{Επιβλέπων:} & Χρήστος Βόλος\\
		& Καθηγητής Α.Π.Θ.\\
	\end{tabular}
	
	\centering
	\vspace{\fill}
	\today
	
\end{titlepage}

\begin{abstract}
	H παρούσα πτυχιακή εργασία  αρχικά μελετάει παραλλαγές γνωστών μη - γραμμικών διακριτών δυναμικών συστημάτων που εμφανίζουν χαοτικη συμπεριφορά και αναλύει τα φαινόμενα που παρατηρούνται με την μεταβολή διάφορων παραμέτρων.Σε επόμενο χρόνο ελέγχεται η δυνατότητα τους να χρησιμοποιηθούν για την μελέτη ενός ρομποτικού συστήματος και εκτίμηση της αποτελεσματικότητας αυτών.
	
	Η ανάλυση χωρίζεται σε πέντε κεφάλαια.
	
	Στο πρώτο κεφάλαιο αναπτύσσεται  το απαραίτητο θεωρητικό υπόβαθρο για την κατανόηση της μελέτης των συστημάτων. Ειδικότερα αναλύεται ο ορισμός των δυναμικών συστηματων και η σχέση τους με το χάος. Ορίζονται τα εργαλεία που αξιοποιούνται στην παρούσα εργασία για την μελέτη των συστημάτων, όπως και τα φαινόμενα που παρατηρήθηκαν μέσα σπο αυτήν.
	
	
	Στα επόμενα τρία κεφάλαια μελετούνται οι παραλλαγες τριών μη - γραμμικών διακριτών δυναμικών συστημάτων. Συγκεκριμένα στο δεύτερο κεφάλαιο μελετάται η παραλλαγή του \emph{λογιστικού Χάρτη} και οι συμπεριφορές που εμφανίζει στο διάγραμμα διακλάδωσης και Lyapunov όσο μεταβάλλεται μία παράμετρος. Στο τρίτο κεφάλαιο μελετάται η παραλλαγή του \emph{sine-sinh-sine Χάρτη} και οι συμπεριφορές που εμφανίζει όσο μεταβάλλεται μία παράμετρος. Τέλος συμβαίνει το ίδιο και για τέταρτο κεφάλαιο για την παραλλαγή του  \emph{Chebysev Χάρτη}.
	
	Στο πέμπτο κεφάλαιο μελετήθηκε η κίνηση του ρομποτικού συστήματος και η κάλυψη μιας συγκεκριμένης περιοχής του χώρου, με την αξιοποίηση του \emph{λογιστικού Χάρτη} συναρτήσει διαφόρων παραμέτρων.Στην αρχή του κεφαλαίου  αναπτύσσεται μαθηματικά η μελέτη του ρομποτικού συστήματος.
	
	
\end{abstract}

\begin{otherlanguage}{english} 
\begin{abstract}
	The following thesis studies variations of known non-linear discrete dynamic systems that display chaotic behavior and analyzes the phenomena observed with the change of various parameters. Next it analyzes the possibility to be used for the study of a robotic system and evaluates their effectiveness.
	
	The theis is divided into five chapters.
	
	The first chapter develops the necessary theoretical background for understanding non-linear discrete dynamic systems. Specifically, the definition of dynamic systems and their relationship with chaos is analyzed . The tools used in this work to study the systems are defined, as well as the phenomena observed in it.
	
	In the next three chapters, the variants of three non-linear discrete dynamical systems are studied. Specifically, in the second chapter, the variant of the \emph{Logistics Map} is studied and the behaviors it displays in the bifurcation and Lyapunov diagram as one parameter changes. In the third chapter, the variation of the \emph{sine-sinh-sine Map} is studied and the behaviors it displays as one parameter changes aswell. Finally, the same thing happens for the fourth chapter for the variation of \emph{Chebysev Map}.
	
	In the fifth chapter, the movement of the robotic system and the coverage of a specific area of ​​space was studied, with the utilization of the \emph{Logistic Map} as a function of various parameters. At the beginning of the chapter, the study of the robotic system is developed mathematically.
	
	
\end{abstract}
\end{otherlanguage}
\thispagestyle{empty}


\section*{Ευχαριστίες}
\thispagestyle{empty}

Θα ήθελα να ευχαριστήσω τον επιβλέποντα καθηγητή κ. Χρήστο Βόλο για τον χρόνο που αφιέρωσε για να μου απαντήσει σε ό,τι απορία είχα γύρω απο το θέμα της πτυχιακής εργασίας , αλλά και την υπομονή που έδειξε μέσα σε αυτό το χρονικό διάστημα , μετατρέποντας την εργασία σε μία ευχάριστη εμπειρία.

Επίσης θα ήθελα να ευχαριστήσω δύο κοντινά μου άτομα για την υπομονή που δείξανε και τον χρόνο που αφιέρωσαν στο να με βοηθήσουν στον κώδικα που έγραψα, όπως και τους ανθρώπους του εργαστηρίου Lanscom που ήταν εκεί για να απαντήσουν κάθε μου ερώτηση.

Τέλος θα ήθελα να ευχαριστήσω τους κοντινούς μου ανθρώπους, που χωρίς την συμπαράσταση τους, τα θερμά τους λόγια, και τις στιγμές που πίστευαν περισσότερο αυτοί σε εμένα, δεν θα μπορούσα να τελειώσω αυτή την εργασία.

\clearpage
